\documentclass[12pt]{article}
\usepackage{amsmath} \usepackage{amssymb} \usepackage{amsthm}
% These three packages are from the American Mathematical Society and includes
% all of the important symbols and operations 
\usepackage{fullpage}
% By default, an article has some vary large margins to fit the smaller page
% format.  This allows us to use more standard margins.

\usepackage{systeme}

\usepackage{blindtext}

\setlength{\parskip}{1em}
% This gives us a full line break when we write a new paragraph

\title{MATH 308 Conceptual Problems, Week 4 \vspace{1em}\\\normalsize{SUPPLEMENTARY CONCEPTUAL PROBLEMS, 8d \& 8e}}
\author{Jai Broome}
\date{28 January 2020}

\begin{document}
% Once we have all of our packages and setting announced, we need to begin our
% document.  You will notice that at the end of the writing there is an end
% document statements.  Many options use this begin and end syntax.

\maketitle

\section{Question prompt}

In each of the following, $\bar{x}$ is the vector whose components are the coefficients
in $f(t)$, and $\bar{y}$ is the vector whose components are the coefficients in $f'(t)$. Taking
the derivative of $f(t)$ in the usual way, you can find formulas for the $y$'s in terms
of the $x$'s. Then find a matrix $A$ such that $\bar{y}=A\bar{x}$.

\section{Reasoning and motivation}

A few things have to be true for us to approach these derivave problems using the linear algebra
tools that we've learned so far:

\begin{itemize}
  \item We need to be able to differentiate the general form our function $f(t)$ 
    % general in the sense that it has the x's
  \item Our function has to be of the form $f(t)=x_1g_1(t)+x_2g_2(t)+\cdots +x_{n}g_n(t)$ and
    our derivative has to be of the form $f(t)=y_1g_1(t)+y_2g_2(t)+\cdots +y_{n}g_n(t)$
    % some of these X_is can be zero
\end{itemize}

This second point means that the plot of $f'(t)$ will have some visual similarity to the plot of
$f(t)$, since they're both composed of a series of functions (the $g_i(t)$s) multiplied by constants.
We'll see this when we look at the plots in 8d and 8e.

An important thing to remember is that these one-dimensional functions $f(t)$ are mappings from
$\mathbb{R} \rightarrow \mathbb{R}$; the function takes one value $t$ and returns one value $f(t)$.
\textit{However}, our task here is to find a mapping $\mathbb{R}^n \rightarrow \mathbb{R}^n$ which
takes our n-dimensional vector $\bar{x}$ and produces an n-dimensional vector $\bar{y}$.
% this is our matrix A, right?

\section{8d}
Question prompt: $f(t) = x_1\sin(t) + x_2\cos(t) + x_3t\sin(t) + x_4t\cos(t)$,
and $f'(t)$ has the same form with coefficients $y_1$, $y_2$, $y_3$, $y_4$.

Find the derivative the usual way, treating the $x_i$s as constants, and making use of the product
rule: \\

% Add intermediate steps:
\begin{equation}
f'(t) = (-x_2 + x_3)\sin(t) + (x_1 + x_4)\cos(t) + (-x_4)t\sin(t) + (x_3)t\cos(t)
\end{equation}

Define the $y_i$s in $f'(t)$ so that they correspond to same positions as the $x_i$s in $f(t)$:
% that is, y_1 is in front sin(t) etc

\[
\systeme{-x_2 + x_3 = y_1, x_1 + x_4 = y_2, -x_4 = y_3,x_3 = y_4}
\]

In matrix form, this is our solution $A$:

\[
\begin{vmatrix}
  0 & -1 & 1 & 0 \\
  1 & 0 & 0 & 1 \\
  0 & 0 & 0 & -1 \\
  0 & 0 & 1 & 0 
\end{vmatrix}
\]

We can check our work by seeing if we get the same solution by plugging
$\bar{x}$ into $f(t)$ and differentiating; and computing $A\bar{x}$. 
% make sure you remember the rules for matrix multiplication.
\[
  \frac{d}{dt}(2\sin(t)+4\cos(t)+6t\sin(t)+8t\cos(t)) = 
\]
\[
  2\sin(t)+10\cos(t)-8t\sin(t)+6t\cos(t)
\]
\bigbreak{}
\[
  \bar{y} = A\bar{x} = 
  \begin{vmatrix}
    0 & -1 & 1 & 0 \\
    1 & 0 & 0 & 1 \\
    0 & 0 & 0 & -1 \\
    0 & 0 & 1 & 0 
    \end{vmatrix}
  \begin{vmatrix} 2 \\ 4 \\ 6 \\ 8\end{vmatrix} = 
  \begin{vmatrix} 2 \\ 10 \\ -8 \\ 6 \end{vmatrix}
\]

\section{8e}

Question prompt: $f(t) = (x_1 + x_2t + x_3t^2)e^{-kt}$, and $f'(t)$ has the same
form with coefficients $y_1$, $y_2$, $y_3$.

\[
  f'(t) = (-kx_1 + x_2 + (-kx_2+2x_3)t - kx_3t^2)e^{-kt}
\]
\[
  % systeme is grouping on the ks, need to sort this out
  \systeme{-kx_1 + x_2 = y_1, -kx_2+2x_3 = y_2, -kx_{3} = y_3}
\]
\[
  \begin{vmatrix}
    -k & 1 & 0 \\
    0 & -k & 2 \\
    0 & 0 & -k
  \end{vmatrix}
\]
\[
  \frac{d}{dt}((5+10t+15t^2)e^{-kt}) = 
\]
\[
  (10-5k + (30-10k)t -15kt^2)e^{-2t}
\]
\medbreak{}
\[
  \bar{y} = A\bar{x} = 
  \begin{vmatrix}
    -k & 1 & 0 \\
    0 & -k & 2 \\
    0 & 0 & -k
  \end{vmatrix}
  \begin{vmatrix} 5 \\ 10 \\ 15 \end{vmatrix} = 
  \begin{vmatrix} 10-5k \\ 30-10k \\ -15k \end{vmatrix}
\]
\end{document}
