\documentclass[12pt]{article}
\usepackage{amsmath} \usepackage{amssymb} \usepackage{amsthm}
% These three packages are from the American Mathematical Society and includes
% all of the important symbols and operations 
\usepackage{fullpage}
% By default, an article has some vary large margins to fit the smaller page
% format.  This allows us to use more standard margins.

\usepackage{blindtext}

\setlength{\parskip}{1em}
% This gives us a full line break when we write a new paragraph

\title{MATH 308 Conceptual Problems, Week 4 \vspace{1em}\\\normalsize{SUPPLEMENTARY CONCEPTUAL PROBLEMS, 8d \& 8e}}
\author{Jai Broome}
\date{28 January 2020}

\begin{document}
% Once we have all of our packages and setting announced, we need to begin our
% document.  You will notice that at the end of the writing there is an end
% document statements.  Many options use this begin and end syntax.

\maketitle

\section{Question prompt}

In each of the following, $\bar{x}$ is the vector whose components are the coefficients
in $f(t)$, and $\bar{y}$ is the vector whose components are the coefficients in $f'(t)$. Taking
the derivative of $f(t)$ in the usual way, you can find formulas for the $y$'s in terms
of the $x$'s. Then find a matrix $A$ such that $\bar{y}=A\bar{x}$.

\section{Reasoning and motivation}

A few things have to be true for us to approach these derivave problems using the linear algebra
tools that we've learned so far:

\begin{itemize}
  \item We need to be able to differentiate the general form our function $f(t)$ 
    % general in the sense that it has the x's
  \item Our function has to be of the form $f(t)=x_1g_1(t)+x_2g_2(t)+\cdots +x_ng_n(t)$ and
    our derivative has to be of the form $f(t)=y_1g_1(t)+y_2g_2(t)+\cdots +y_ng_n(t)$
    % some of these X_is can be zero
\end{itemize}

\end{document}
