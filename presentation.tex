\documentclass[12pt]{article}
\usepackage{amsmath}
\usepackage{amssymb}
\usepackage{amsthm}
\usepackage{fullpage}
\usepackage{systeme}
\usepackage{blindtext}
\usepackage{graphicx}
\graphicspath{{./fig/}}
\setlength{\parskip}{1em}
% This gives us a full line break when we write a new paragraph

\title{MATH 308 Conceptual Problems, Week 4 \vspace{1em}\\\normalsize{SUPPLEMENTARY CONCEPTUAL PROBLEMS, 8d \& 8e}}
\author{Jai Broome}
\date{28 January 2020}

\begin{document}

\maketitle

\section{Question prompt}

In each of the following, $\bar{x}$ is the vector whose components are the coefficients
in $f(t)$, and $\bar{y}$ is the vector whose components are the coefficients in $f'(t)$. Taking
the derivative of $f(t)$ in the usual way, you can find formulas for the $y$'s in terms
of the $x$'s. Then find a matrix $A$ such that $\bar{y}=A\bar{x}$.

\section{8d}
\textit{Question prompt}: $f(t) = x_1\sin(t) + x_2\cos(t) + x_3t\sin(t) + x_4t\cos(t)$,
and $f'(t)$ has the same form with coefficients $y_1$, $y_2$, $y_3$, $y_4$.

The reason we can approach this problem this way is because
\begin{itemize}
  \item we can differentiate $f(t)$ in its general form, 
    % that is, we don't need to know specific values for \bar{x}
  \item and because the prompt tells us that $f'(t)$ will have a $\sin(t)$ term,
    a $\cos(t)$ term, a $t\sin(t)$ term and a $t\cos(t)$ term.
\end{itemize}
% could be any function g_i(t) that appears in the function and derivative
% could be zeroes
% and because we can tell by looking at each term

This second point means that the plot of $f'(t)$ will have some visual
similarity to the plot of $f(t)$, since they're both composed of a series of
the same functions functions multiplied by constants. \\

The tricky thing about this type of problem is that $f(t)$ and it's derivatives are 1-dimensional
mappings $\mathbb{R} \rightarrow \mathbb{R}$; \textit{however}, our goal here is to find a mapping
$\mathbb{R}^4 \rightarrow \mathbb{R}^4$ that takes our 4-dimensional $\bar{x}$ and produces a
4-dimensional $\bar{y}$
% the function takes one value $t$ and returns one value $f(t)$.

Find the derivative the usual way, treating the $x_i$s as constants, and making use of the product
rule:

\[
f'(t) = (-x_2 + x_3)\sin(t) + (x_1 + x_4)\cos(t) + (-x_4)t\sin(t) + (x_3)t\cos(t)
\]

Let's pick a vector $\bar{x}=(2,4,6,8)$ to visualize $f(t)$ and $f'(t)$, and to later check our
work. Inserting our values for the $x_i$s:

\includegraphics[width=8cm]{plot_8d_f} \includegraphics[width=8cm]{plot_8d_derivative} \\
\includegraphics[width=17cm]{plot_8d_both} \\
\newpage
Define the $y_i$s in $f'(t)$ so that they correspond to same positions as the $x_i$s in $f(t)$:
% that is, y_1 is in front sin(t) etc

\[
\systeme{-x_2 + x_3 = y_1, x_1 + x_4 = y_2, -x_4 = y_3,x_3 = y_4}
\]

In matrix form, this is our solution $A$:

\[
\begin{bmatrix}
  0 & -1 & 1 & 0 \\
  1 & 0 & 0 & 1 \\
  0 & 0 & 0 & -1 \\
  0 & 0 & 1 & 0 
\end{bmatrix}
\]

We can check our work by seeing if we get the same solution by plugging
$\bar{x}$ into $f(t)$ and differentiating; and computing $A\bar{x}$. 
\[
  \frac{d}{dt}(2\sin(t)+4\cos(t)+6t\sin(t)+8t\cos(t)) = 
\]
\[
  2\sin(t)+10\cos(t)-8t\sin(t)+6t\cos(t)
\]
\bigbreak{}
\[
  \bar{y} = A\bar{x} = 
  \begin{bmatrix}
    0 & -1 & 1 & 0 \\
    1 & 0 & 0 & 1 \\
    0 & 0 & 0 & -1 \\
    0 & 0 & 1 & 0 
  \end{bmatrix}
  \begin{bmatrix} 2 \\ 4 \\ 6 \\ 8\end{bmatrix} = 
  \begin{bmatrix} 2 \\ 10 \\ -8 \\ 6 \end{bmatrix}
\]

\newpage

Briefly, this lets us quickly calculate $f''(t)$. Let $\bar{z}$ be the coefficients in $f''(t)$.

\[
  \bar{z} = A\bar{y} = 
  \begin{bmatrix}
    0 & -1 & 1 & 0 \\
    1 & 0 & 0 & 1 \\
    0 & 0 & 0 & -1 \\
    0 & 0 & 1 & 0 
  \end{bmatrix}
  \begin{bmatrix} 2 \\ 10 \\ -8 \\ 6 \end{bmatrix} = 
  \begin{bmatrix} -18 \\ 8 \\ -6 \\ -8 \end{bmatrix}
\]
\[
  \rightarrow f''(t) = -18\sin(t) + 8\cos(t) - 6t\sin(t) - 8t\cos(t)
  % all that entailed was multiplying a 4x4 matrix by a vector
\]

\section{8e}

\textit{Question prompt}: $f(t) = (x_1 + x_2t + x_3t^2)e^{-kt}$, and $f'(t)$ has the same
form with coefficients $y_1$, $y_2$, $y_3$.

As before, our first step is to differentiate the general form of $f(t)$.

\[
  f'(t) = (-kx_1 + x_2 + (-kx_2+2x_3)t - kx_3t^2)e^{-kt}
\]

We have $k$ terms in here, but that's okay. $k$ is a parameter in our equation, not an unkown and
is \textit{not} in $\bar{x}$. It appears in our solution, but that's alright. We can do matrix
multiplication that includes constant terms.

Next, define the $y_i$s so that they correspond to the constant, $t$ and $t^2$ terms

\[
  \systeme{-kx_1 + x_2 = y_1, -kx_2+2x_3 = y_2, -kx_3 = y_3}
\]

In matrix form, this is our solution $A$:

\[
  \begin{bmatrix}
    -k & 1 & 0 \\
    0 & -k & 2 \\
    0 & 0 & -k
  \end{bmatrix}
\]

\newpage

Using $\bar{x} = (5, 10, 15)$, we can compare our solution differentiating $f(t)$ and computing $A\bar{x}$.

\[
  \frac{d}{dt}((5+10t+15t^2)e^{-kt}) = 
\]
\[
  (10-5k + (30-10k)t -15kt^2)e^{-kt}
\]
\medbreak{}
\[
  \bar{y} = A\bar{x} = 
  \begin{bmatrix}
    -k & 1 & 0 \\
    0 & -k & 2 \\
    0 & 0 & -k
  \end{bmatrix}
  \begin{bmatrix} 5 \\ 10 \\ 15 \end{bmatrix} = 
  \begin{bmatrix} 10-5k \\ 30-10k \\ -15k \end{bmatrix}
\]
\bigbreak{}
\bigbreak{}
\bigbreak{}
\bigbreak{}
\bigbreak{}
\bigbreak{}
\bigbreak{}
\bigbreak{}
\bigbreak{}
\bigbreak{}
\bigbreak{}
\bigbreak{}

\textit{presentation materials avalable at} github.com/broomej/m308\_w4\_cp
\end{document}
